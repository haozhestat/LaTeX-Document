%!TEX program = xelatex
\documentclass[a4paper,UTF8]{ctexart}
\usepackage[unicode=true,colorlinks,urlcolor=blue,linkcolor=blue,bookmarksnumbered=true]{hyperref}
\usepackage{latexsym,amssymb,amsmath,amsbsy,amsopn,amstext,amsthm,amsxtra,color,multicol,bm,calc,ifpdf}
\usepackage{graphicx}
\usepackage{diagbox}   % 绘制表格斜线
\usepackage{enumerate}
\usepackage{epstopdf}
\usepackage{fancyhdr}
\usepackage{subfigure}
\usepackage{listings}
\usepackage{multirow}
\usepackage{makeidx}
\usepackage{xcolor} 
\usepackage{fontspec}                            % 建立索引宏包
\graphicspath{{figures/}}  % 设置图片搜索路径
\theoremstyle{plain} \newtheorem{theorem}{定理}[section]
\theoremstyle{plain} \newtheorem{definition}{定义}[section]
\theoremstyle{plain} \newtheorem{lemma}{引理}[section]
\theoremstyle{plain} \newtheorem{proposition}{命题}[section]
\theoremstyle{plain} \newtheorem{example}{例}[section]
\theoremstyle{plain} \newtheorem{remark}{注}[section]
\theoremstyle{plain} \newtheorem{corollary}{推论}[section]
\newfontfamily\courier{Courier New}
\lstset{linewidth=1.1\textwidth,
        numbers=left, %设置行号位置 
        basicstyle=\small\courier,
        numberstyle=\tiny\courier, %设置行号大小  
        keywordstyle=\color{blue}\courier, %设置关键字颜色  
        %identifierstyle=\bf,
        commentstyle=\it\color[cmyk]{1,0,1,0}\courier, %设置注释颜色 
        stringstyle=\it\color[RGB]{128,0,0}\courier,
        %framexleftmargin=10mm,
        frame=single, %设置边框格式  
        backgroundcolor=\color[RGB]{245,245,244},
        %escapeinside=``, %逃逸字符(1左面的键),用于显示中文  
        breaklines, %自动折行  
        extendedchars=false, %解决代码跨页时,章节标题,页眉等汉字不显示的问题  
        xleftmargin=2em,xrightmargin=2em, aboveskip=1em, %设置边距  
        tabsize=4, %设置tab空格数  
        showspaces=false %不显示空格  
        basicstyle=\small\courier
       }  
\newenvironment{mysolution}{{\color{blue} 解}: }{{\color{magenta}\qed}}
\newcommand\diff{\,{\mathrm d}} %定义微分d
\newcommand{\p}[3]{\frac{\partial^{#1}#2}{\partial{#3}^{#1}}}  %定义求偏导算子
\newcommand{\ucite}[1]{\textsuperscript{\cite{#1}}}  %参考文献引用:上标用\ucite{ },文中用\cite{ }

\begin{document}
\title{
\includegraphics[width=0.65\textwidth]{onepiece.pdf}\\
\vspace{2em}
\textbf{隐马尔科夫模型学习笔记}}
\author{\emph{李向阳} \quad {\color{blue} d1142845997@gmail.com} }
\date{}


\maketitle
\thispagestyle{empty}

\newpage


\tableofcontents

\newpage

\section{引入}
隐马尔可夫模型(Hidden Markov Model, HMM)是一个统计模型, 它用来描述一个含有未知参数的马尔科夫过程. 其特点是从可观察到的参数中确定该过程的隐含参数, 然后利用这些参数作进一步的分析.

在正常的马尔可夫模型中, 状态对于观察者来说是直接可见的, 此时状态之间的转移概率便是全部的模型参数. 而在隐马尔可夫模型中, 状态并不是直接可见的, 但受状态影响的某些变量则是可见的. 每一个状态在可能输出的符号上都有一个概率分布, 因此输出符号的序列能够透露出状态序列的一些信息.

本质上, HMM 是描述由隐藏的马尔可夫链随机生成观测序列的过程, 属于生成模型.


\section{模型介绍}
隐马尔可夫模型是关于时序的概率模型, 描述由一个隐藏的马尔可夫链随机生成不可观测的状态随机序列, 再由各个状态生成一个观测而产生观测随机序列的过程. 隐藏的马尔可夫链随机生成的状态的序列称为状态序列(state sequence), 通常假设状态变量是隐藏的、不可观测的. 每个状态生成一个观测, 而由此产生的观测的随机序列, 称为观测序列(observation sequence). 序列的每一个位置又可看做一个时刻. 

下面用记号来描述一下, 对于有略微强迫症的人来说, 记号一直是一件令人蛋疼的事情, 因为很多资料中的记号都不大相同. HMM 模型也是一样, 这里我采用李航《统计学习方法》中的记号, 并尽量与之前的记号保持一致.

设$I$是长度为$T$的状态序列, $O$是对应的观测序列.
\begin{equation*}
I = (i_1, i_2, \cdots, i_T), O = (o_1, o_2, \cdots, o_T)
\end{equation*}

状态序列和观测序列都是有各自的取值空间的, 设$Q$是所有可能的状态的集合, 其元素个数为$N$, 设$V$是所有可能的观测的集合, 其元素个数为$M$.
\begin{equation*}
Q = \{q_1, q_2, \cdots, q_N\}, V = \{v_1, v_2, \cdots, v_M\}
\end{equation*}

设矩阵$A$是状态转移概率矩阵, 即$A = (a_{ij})_{N \times N}$, 其中
\begin{equation*}
a_{ij} = P(i_{t+1} = q_{j} | i_{t} = q_{i}), i, j = 1, 2, \cdots, N
\end{equation*}

也就是说$a_{ij}$表示在时刻$t$处于状态$q_{i}$的条件下在时刻$t+1$转移到状态$q_{j}$的概率.

设矩阵$B$是观测概率矩阵, 即$B = (b_{jk})_{N \times M}$, 其中
\begin{equation*}
b_{jk} = P(o_{t} = v_{k} | i_{t} = q_{j}), j = 1,2, \cdots, N, k = 1,2, \cdots,M
\end{equation*}

也就是说$b_{jk}$表示在时刻$t$处于状态$q_{j}$的条件下生成观测$v_{k}$的概率.

设$\bm{\pi}$是初始状态的概率向量, 即$\bm{\pi} = (\pi_i)$, 其中
\begin{equation*}
\pi_{i} = P(i_{1} = q_{i}), i = 1,2,\cdots,N
\end{equation*}

也就是说$\pi_{i}$表示时刻$t=1$时处于状态$q_{i}$的概率.

隐马尔可夫模型由初始状态概率向量$\bm{\pi}$、状态转移概率矩阵$A$和观测概率矩阵$B$完全决定, 其中$\bm{\pi}$和$A$决定状态序列, $B$决定观测序列, 因此, 隐马尔科夫模型$\lambda$可以用三元符号表示, 即
\begin{equation*}
\lambda = (A, B, \bm{\pi})
\end{equation*}

$A, B, \bm{\pi}$称为隐马尔可夫模型的三要素.

下面举一个具体的例子, 这个知乎上有很多, 可见\url{https://www.zhihu.com/question/20962240}. 这里还是用李航《统计学习方法》上摸球的例子.

假设有$4$个盒子, 每个盒子里都装有红白两种颜色的球, 盒子里的红白球数可见表
\begin{table}[!htb]
\centering
\caption{各盒子的红白球数}
\label{box}
\begin{tabular}{ccccc}
  \hline
    % after \: \hline or \cline{col1-col2} \cline{col3-col4} ...
    \textbf{盒子} & \textbf{ID1} & \textbf{ID2} & \textbf{ID3} & \textbf{ID4} \\
    \hline
    红球数   &  5  &  3  & 6 & 8 \\
    \hline
    白球数   &  5  &  7  & 4 & 2 \\
  \hline
\end{tabular}
\end{table}

按照下面的方法摸球, 产生一个球的颜色的观测序列: 开始, 从$4$个盒子中随机(等概论)的选取一个盒子, 从这个盒子中随机摸出一个球, 记录其颜色后, 放回; 然后, 从当前盒子中转移到下一个盒子, 转移规则是: 如果当前盒子是盒子1, 那么下一个盒子一定是盒子2, 如果当前盒子是盒子2或者3, 那么分别以概率$0.4$和$0.6$转移到左边或右边的盒子, 如果当前是盒子4, 那么各以$0.5$的概率停留在盒子4或转移到盒子3; 确定转移的盒子后, 再从这个盒子中随机摸出一个球, 记录其颜色, 放回; 如此下去, 不断重复. 比如进行了$5$次, 得到了一个球的颜色的观测序列:
\begin{equation*}
O = \{\textrm{红, 红, 白, 白, 红}\}
\end{equation*}

在这个过程中, 假设观察者只能观测到球的颜色的序列, 观测不到球是从哪个盒子中取出来的, 即观测不到盒子的序列.

本例子中, 盒子的序列便是状态序列, 而球的颜色的序列便是观测序列, 前者是隐藏的, 只有后者是可观测的. 根据所给条件, 可以明确状态集合、观测集合、序列长度和模型的三要素.

盒子对应状态, 状态的集合是
\begin{equation*}
Q = \{\textrm{盒子1, 盒子2, 盒子3, 盒子4}\}, N = 4
\end{equation*}

球的颜色对应观测序列, 观测的集合是
\begin{equation*}
V = \{\textrm{红, 白}\}, M = 2
\end{equation*}

本次具体的状态序列和观测序列的长度为$T = 5$, 其中状态序列未知, 而观测序列为
\begin{equation*}
O = \{\textrm{红, 红, 白, 白, 红}\}
\end{equation*}

初始概率分布为
\begin{equation*}
\bm{\pi} = (0.25, 0.25, 0.25, 0.25)^{T}
\end{equation*}

状态转移概率矩阵为
$$
A = 
\begin{bmatrix}
0 & 1 & 0 & 0 \\ 
0.4 & 0 & 0.6 & 0 \\ 
0 & 0.4 & 0 & 0.6 \\ 
0 & 0 & 0.5 & 0.5
\end{bmatrix}
$$

观测概率矩阵为
$$
B = 
\begin{bmatrix}
0.5 & 0.5 \\ 
0.3 & 0.7 \\ 
0.6 & 0.4 \\ 
0.8 & 0.2 
\end{bmatrix}
$$

这便是本例子的模型三要素.

那么隐马尔可夫模型有什么用呢? 它主要解决$3$类基本问题.
\begin{enumerate}[(1)]
\item 解码问题, 也称为预测问题. 给定模型$\lambda = (A, B, \bm{\pi})$和观测序列, 求最有可能的状态序列. 

这个问题有两种解法, 会给出两个不同的答案. 每个答案都合理, 因为答案的意义不太一样.
第一种解法是求最大似然路径, 就是求一串状态序列, 使得其产生观测结果序列的概率最大. 第二种解法, 就不是求一组状态序列了, 而是求出每次的观测结果分别是某种状态产生的概率.

\item 概率计算问题. 给定模型$\lambda = (A, B, \bm{\pi})$和观测序列, 计算在模型$\lambda$下观测序列出现的概率.

这个问题的意义看似不大, 毕竟我们的观测结果很多时候都对应了一个比较大的概率. 这个问题的目的在于, 检测观测到的结果和已知的模型是否吻合. 如果很多次结果都对应了比较小的概率, 那么就说明我们已知的模型很有可能是错的.

\item 学习问题. 已知观测序列, 要估计出模型$\lambda = (A, B, \bm{\pi})$的参数.

这个问题就很重要了, 因为这是最常见的情况. 很多时候我们只有可见结果, 不知道 HMM 模型里面的参数, 我们需要从可见结果中估计出这些参数.

\end{enumerate}









\section{关于 HMM 的补充}












\section{总结}
\subsection{参考资料}
\begin{enumerate}[(1)]
\item 李航的《统计学习方法》

\item 周志华的《机器学习》

\item 维基: \url{https://zh.wikipedia.org/wiki/%E9%9A%90%E9%A9%AC%E5%B0%94%E5%8F%AF%E5%A4%AB%E6%A8%A1%E5%9E%8B}

\item 知乎: \url{https://www.zhihu.com/question/20962240}, 有非常形象的例子.



\end{enumerate}





\begin{thebibliography}{4}
  \bibitem{1} 李荣华.\emph{偏微分方程数值解法}.高等教育出版社(2010) 
  \bibitem{2} Zhilin Li,Zhonghua Qiao,Tao Tang.\emph{Numerical Solutions of Partial Differential Equations-An Introduction to Finite Difference and Finite Element Methods}.(2011)
  \bibitem{3} 孙志忠.\emph{偏微分方程数值解法}.科学出版社(2011)
  \bibitem{4} 陆金甫 关治.\emph{偏微分方程数值解法}.清华大学出版社(2004)
  
\end{thebibliography}

\newpage

\section*{附录}








\end{document}